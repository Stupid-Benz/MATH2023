\documentclass{book}

\usepackage{xcolor} % define colors
\usepackage{amsthm} % for theorem environments
\usepackage{amssymb} % for math symbols
\usepackage{amsmath} % for advanced math typesetting
\usepackage[margin=1in]{geometry} % set page margins
\usepackage{tikz, tikz-3dplot} % for 2D and 3D graphics
\usepackage[pdfborder={0 0 0}]{hyperref} % for hyperlinks in the document %removed the red box around links
\usepackage{subcaption} % for subcaption


\usepackage{pgfplots}
\pgfplotsset{compat=1.15}


\definecolor{ocre}{RGB}{0,83,166}

\newtheorem{example}{Example}[chapter]
% Non-italic
\theoremstyle{remark}
\newtheorem*{solution}{Solution}
\newtheorem*{remark}{Remark}

%Added some common sets
\newcommand{\N}{\mathbb{N}}
\newcommand{\Z}{\mathbb{Z}}
\newcommand{\Q}{\mathbb{Q}}
\newcommand{\R}{\mathbb{R}}
\newcommand{\C}{\mathbb{C}}


\begin{document}

\frontmatter
\title{\sffamily\Huge\color{ocre} Multivariable Calculus}
\author{\sffamily Department of Mathematics\\\sffamily Hong Kong University of Science and Technology}
\date{\sffamily \today}
\maketitle

\tableofcontents

\mainmatter

\chapter{Vectors and the Geometry of Space}

\section{Three-Dimensional Coordinate Systems}

We would use an ordered tuple of three numbers $(x, y, z)$ to represent a point in three-dimensional space. The three numbers correspond to the distances along the $x$-axis, $y$-axis, and $z$-axis respectively.

Moreover, we can use a vector to represent a point in space. A vector $\mathbf{v}$ can be expressed as:
\[
    \mathbf{v} = \langle x, y, z \rangle = x\mathbf{i} + y\mathbf{j} + z\mathbf{k}
\]
where $\mathbf{i}$, $\mathbf{j}$, and $\mathbf{k}$ are the unit vectors along the $x$-, $y$-, and $z$-axes respectively.

\begin{remark}
    Unit vectors are vectors with a magnitude of 1. They are often used to indicate direction.
\end{remark}

The distance, or norm, of the vector $\mathbf{v}$ from the origin can be calculated using the formula:
\[
    \|\mathbf{v}\|_2 = \|\mathbf{v}\| = \sqrt{x^2 + y^2 + z^2}
\]
This is also known as the Euclidean norm.

As we are used to consider two-dimensional planes, we always consider the following equations as circles in two-dimensional space:
\[
    x^2 + y^2 = r^2
\]
However, in three-dimensional space, this equation represents a cylinder extending infinitely along the $z$-axis. As implicitly, the equation does not restrict the value of $z$. Then the set of points satisfying the equation forms a cylinder.

In two-dimensional case, the set of points satisfying the equation $x^2 + y^2 = r^2$ represents a circle of radius $r$ centered at the origin:
\[
    S^1 = \{ (x, y) \mid x^2 + y^2 = r^2 \}
\]
In three-dimensional case, the set of points satisfying the equation $x^2 + y^2 = r^2$ represents a cylinder of radius $r$ centered along the $z$-axis:
\[
    C = \{ (x, y, z) \mid x^2 + y^2 = r^2, z \in \mathbb{R} \}
\]

So if we want to represent a two-dimensional circle in three-dimensional space, we need to add an additional constraint on $z$. For example, the set of points satisfying the equations $x^2 + y^2 = r^2$ and $z = 0$ represents a circle of radius $r$ in the $xy$-plane:
\[
    S^1 = \{ (x, y, z) \mid x^2 + y^2 = r^2, z = 0 \}
\]

For vector operations, we have:
\begin{itemize}
    \item Vector Addition: $\mathbf{a} + \mathbf{b} = \langle a_1 + b_1, a_2 + b_2, a_3 + b_3 \rangle$
    \item Scalar Multiplication: $c\mathbf{a} = \langle ca_1, ca_2, ca_3 \rangle$
\end{itemize}

Also, we have the dot product and cross product defined as:
\begin{itemize}
    \item Dot Product: $\mathbf{a} \cdot \mathbf{b} = a_1b_1 + a_2b_2 + a_3b_3$
    \item Cross Product: $\mathbf{a} \times \mathbf{b} = \langle a_2b_3 - a_3b_2, a_3b_1 - a_1b_3, a_1b_2 - a_2b_1 \rangle$
\end{itemize}

Moreover, the dot product can also be expressed in terms of the magnitudes of the vectors and the angle $\theta$ between them:
\[
    \mathbf{a} \cdot \mathbf{b} = \|\mathbf{a}\| \|\mathbf{b}\| \cos\theta
\]
and the magnitude of the cross product can be expressed as:
\[
    \|\mathbf{a} \times \mathbf{b}\| = \|\mathbf{a}\| \|\mathbf{b}\| \sin\theta
\]
It represents the area of the parallelogram formed by the two vectors.

If we want to project vector $\mathbf{b}$ onto vector $\mathbf{a}$, we can use the formula:
\[
    \text{proj}_{\mathbf{a}} \mathbf{b} = \left(\frac{\mathbf{a} \cdot \mathbf{b}}{\|\mathbf{a}\|}\right) \frac{\mathbf{a}}{\|\mathbf{a}\|} = \frac{\mathbf{a} \cdot \mathbf{b}}{\|\mathbf{a}\|^2} \mathbf{a}
\]
The scalar projection of $\mathbf{b}$ onto $\mathbf{a}$ is given by:
\[
    \text{comp}_{\mathbf{a}} \mathbf{b} = \|\mathbf{b}\| \cos\theta = \frac{\mathbf{a} \cdot \mathbf{b}}{\|\mathbf{a}\|}
\]

For the cross product, we can use the following determinant form:
\[
    \mathbf{a} \times \mathbf{b} = \begin{vmatrix}
    \mathbf{i} & \mathbf{j} & \mathbf{k} \\
    a_1 & a_2 & a_3 \\
    b_1 & b_2 & b_3
    \end{vmatrix}
\]

\begin{remark}
    The cross product of two vectors results in a vector that is orthogonal (perpendicular) to both original vectors. The direction of the resulting vector is determined by the right-hand rule.
\end{remark}

\section{Lines and Planes}

\subsection{Lines}

To represent a line in three-dimensional space, we can use a point and a direction vector. If we have a point $P_0(x_0, y_0, z_0)$ on the line and a direction vector $\mathbf{v} = \langle v_1, v_2, v_3 \rangle$, then any point $P(x, y, z)$ on the line, the vector $\overrightarrow{P_0P}$ is parallel to $\mathbf{v}$, i.e., $\overrightarrow{P_0P} = t\mathbf{v}$ for some scalar $t$. Then we have the parametric equations of the line as:
\[
    \langle x, y, z \rangle - \langle x_0, y_0, z_0 \rangle = t \langle v_1, v_2, v_3 \rangle
\]
or equivalently,
\[
    \begin{cases}
        x = x_0 + tv_1 \\
        y = y_0 + tv_2 \\
        z = z_0 + tv_3
    \end{cases}
\]
which are called the \emph{parametric equations} of the line. The $t$ is called the \emph{parameter} of the line.

To visualize the parametric equation of a line in 3D, consider Figure \ref{fig:3d_line_parametric} below.
\begin{figure}[h]
    \centering
    \tdplotsetmaincoords{75}{135}
    \begin{tikzpicture}[scale=1.5, tdplot_main_coords, axis/.style={->,black,thick}, vector/.style={-latex,black,very thick}, vector guide/.style={-latex,thick}, vector extension/.style={densely dashed,red,-latex}]

        %standard tikz coordinate definition using x, y, z coords
        \coordinate (origin) at (0,0,0);
        \coordinate (a) at (0,1,2);
        \coordinate (b) at (-3,0,1);
        %draw axes
        \draw[axis] (0,0,0) -- (4,0,0) node[anchor=north east]{$x$};
        \draw[axis] (0,0,0) -- (0,4,0) node[anchor=north west]{$y$};
        \draw[axis] (0,0,0) -- (0,0,4) node[anchor=south]{$z$};

        % Draw two points
        \draw[fill=black] (a) circle[radius=1pt] node[anchor=south, yshift=0.5em, xshift=0.5em]{$P_0(x_0, y_0, z_0)$};
        \draw[fill=black] (b) circle[radius=1pt] node[anchor=south west]{$P(x, y, z)$};

        %draw guide lines to components
        \draw[vector guide, magenta] (origin) -- (a) node[midway, right, black]{$\mathbf{r}_0$};
        \draw[vector guide, ocre] (origin) -- (b) node[midway, right, black, xshift=0.5em]{$\mathbf{r}(t) = \mathbf{r}_0 + t\mathbf{v}$};

        % Draw parametric lines
        \draw ($(a)+2*($(a)-(b)$)$) node[anchor=south west]{$L$} -- (a) -- (b) -- ($(b)+2*($(b)-(a)$)$);
        \draw[vector guide, orange] (a) -- (b) node[midway, above, black]{$t \mathbf{v}$};
    \end{tikzpicture}
    \caption{Parametric Equation of a Line in 3D}\label{fig:3d_line_parametric}
\end{figure}

From Figure \ref{fig:3d_line_parametric}, we can also write the parametric equations as:
\[
    \mathbf{r}(t) = \overrightarrow{OP_0} + t\mathbf{v}
\]
which is called the \emph{vector form} of the line.

If $\mathbf{v} = \langle v_1, v_2, v_3 \rangle$ where none of $v_1, v_2, v_3$ is zero, we can also express the line in \emph{symmetric form} as:
\[
    \frac{x - x_0}{v_1} = \frac{y - y_0}{v_2} = \frac{z - z_0}{v_3}
\]

\begin{example}
    Find the parametric equations of the line that passes through the points $A(1, 2, 3)$ and $B(4, 5, 6)$. Express the line in vector form, parametric form and symmetric forms.
\end{example}
\begin{solution}
    In order to find the equation of the line, we need 
    \begin{itemize}
        \item A point on the line: $A(1, 2, 3)$;
        \item A direction vector: $\mathbf{v} = \overrightarrow{AB} = \langle 4 - 1, 5 - 2, 6 - 3 \rangle = \langle 3, 3, 3 \rangle$.
    \end{itemize}
    Therefore, the vector form of the line is:
    \[
        \mathbf{r}(t) = \langle 1, 2, 3 \rangle + t \langle 3, 3, 3 \rangle
    \]
    The parametric form of the line is:
    \[
        \begin{cases}
            x = 1 + 3t \\
            y = 2 + 3t \\
            z = 3 + 3t
        \end{cases}
    \]
    The symmetric form of the line is:
    \[
        \frac{x - 1}{3} = \frac{y - 2}{3} = \frac{z - 3}{3}
    \]
\end{solution}

\subsection{Planes}

A plane in three-dimensional space can be defined using a point and a normal vector. If we have a point $P_0(x_0, y_0, z_0)$ on the plane and a normal vector $\mathbf{n} = \langle A, B, C \rangle$, then any point $P(x, y, z)$ on the plane satisfies the condition that the vector $\overrightarrow{P_0P}$ is orthogonal to the normal vector $\mathbf{n}$, i.e., $\mathbf{n} \cdot \overrightarrow{P_0P} = 0$. This leads to the equation of the plane:
\[
    \langle A, B, C \rangle \cdot (\langle x, y, z \rangle - \langle x_0, y_0, z_0 \rangle) = 0
\]
or equivalently,
\[
    A(x - x_0) + B(y - y_0) + C(z - z_0) = 0
\]
which is called the \emph{scalar equation} of the plane. 

Expanding this, we get:
\[
    Ax + By + Cz = Ax_0 + By_0 + Cz_0
\]
or equivalently,
\[
    Ax + By + Cz + D = 0
\]
where $D = -(Ax_0 + By_0 + Cz_0)$ is a constant. It is called a \emph{linear equation} in $x$, $y$ and $z$.

To visualize the equation of a plane in 3D, consider Figure \ref{fig:3d_plane_equation} below.
\begin{figure}[h]
    \centering
    \tdplotsetmaincoords{75}{135}
    \begin{tikzpicture}[x={(1cm,0.4cm)}, y={(8mm, -3mm)}, z={(0cm,1cm)}, line cap=round, line join=round]
        %Coordinates
        %Plane Vertex Points
        \coordinate (x1) at (-2,2,3);
        \coordinate (x2) at (2,2,5);
        \coordinate (x3) at (2,-2,5);
        \coordinate (x4) at (-2,-2,3);
        %Vectors Parallel to Plane
        \coordinate (n1) at ($(x2) - (x1)$);
        \coordinate (n2) at ($(x2) - (x3)$); 
        %Points on Plane
        \coordinate (x5) at ($(x1) + 0.04*(n1) - 0.9*(n2)$);
        \node[outer sep = 1pt, inner sep = 1pt] (x6) at ($(x1) + 0.7*(n1) - 0.3*(n2)$) {};
        \coordinate (x7) at ($(x1) + 0.5*(n1) - 0.5*(n2)$);
        %Beginning of Axis
        \coordinate (O) at (0,0,0);
        %Random Point
        \node[outer sep = 1pt, inner sep = 1pt] (P) at ($(x1) + 0.2*(n1) - 0.4*(n2)$) {};
        
        %Axis     
        \draw[-latex] (-2.5,0,0) -- (2.5,0,0) node[pos = 1.05] {$x$};
        \draw[-latex] (0,-3.5,0) -- (0,3.5,0) node[pos = 1.05] {$y$};
        \draw[-latex] (0,0,0) -- (0,0,7) node[pos = 1.05] {$z$};
        \draw[draw=black, fill=black] (O) circle (1pt) node[below] {${O}$};
        
        %Point on Plane
        \draw[-latex, thick] (O) -- (x6) node[pos=0.45, shift={(0.1,0.3)}] {$\mathbf{r}_0$};
        \draw[-latex, thick] (O) -- (P) node[pos=0.45, shift={(-0.1,-0.3)}] {$\mathbf{r}$};
        %Plane
        \path[draw=black, fill=black!20, thick, opacity = 0.8] (x1) -- (x2) -- (x3) -- (x4) -- (x1);
        \node[shift={(-0.45,0.6)}] at (x3) {$\Pi$};
        %Perpendicular Vector
        \coordinate (direction_vec) at ($(-8,0,24) - (x5)$);

        % Perpendicular Vector at x5
        \draw[-latex, thick] (x5) -- ($(x5) + 0.07*(direction_vec)$) node[pos=0.5, shift={(-0.2,-0.1)}] {$\mathbf{n}$};
        
        % Parallel Vector at x6
        % Now add the scaled vector to x6
        \draw[-latex, thick] (x6) -- ($(x6) + 0.07*(direction_vec)$) node[pos=0.5, shift={(0.2,0.1)}] {$\mathbf{n}$};
        
        %Point on Plane    
        \draw[draw=black, fill=black] (x6) circle (1pt) node[above right] {${P}_0$};
        
        %Z-axis Section
        \draw[draw=black, fill=black] (x7) circle (0.5pt);
        \draw (x7) -- (0,0,6.5);
        
        %Random Point
        \draw[draw=black, fill=black] (P) circle (1pt) node[above left] {$P$};
        \draw[-latex, thick, ocre] (x6) -- (P) node[pos=0.2, above left] {$\overrightarrow{P_0P}$};
    \end{tikzpicture}
    \caption{Equation of a Plane in 3D}\label{fig:3d_plane_equation}
\end{figure}

In order to find $\mathbf{n}$, we can use the cross product.
\begin{example}
    Find the equation of the plane that passes through the points:
    \[
        A(1, 2, 3), \quad B(4, 5, 6), \quad C(7, 8, 0).
    \]
\end{example}
\begin{solution}
    In order to find the equation of the plane, we need
    \begin{itemize}
        \item A point on the plane: $A(1, 2, 3)$;
        \item A normal vector: $\mathbf{n} = \overrightarrow{AB} \times \overrightarrow{AC}$.
    \end{itemize}
    First, we calculate the vectors $\overrightarrow{AB}$ and $\overrightarrow{AC}$:
    \[
        \begin{split}
            \overrightarrow{AB} &= \langle 4 - 1, 5 - 2, 6 - 3 \rangle = \langle 3, 3, 3 \rangle, \\
            \overrightarrow{AC} &= \langle 7 - 1, 8 - 2, 0 - 3 \rangle = \langle 6, 6, -3 \rangle.
        \end{split}
    \]
    Taking the cross product, we have:
    \[
        \overrightarrow{AB} \times \overrightarrow{AC} = \begin{vmatrix}
        \mathbf{i} & \mathbf{j} & \mathbf{k} \\
        3 & 3 & 3 \\
        6 & 6 & -3
        \end{vmatrix} = \langle 0, 0, -9 \rangle.
    \]
    For simplicity, we can take the normal vector as $\mathbf{n} = \langle 0, 0, 1 \rangle$. Therefore, the equation of the plane is:
    \[
        \begin{split}
            0(x - 1) + 0(y - 2) + 1(z - 3) &= 0 \\
            z - 3 &= 0 \\
            z &= 3.
        \end{split}
    \]
\end{solution}

If we have a point $P_1(x_1, y_1, z_1)$ not on the plane, we can calculate the distance from the point to the plane using the formula:
\[
    \text{Distance} = \frac{\| \mathbf{n} \cdot \mathbf{b} \|}{\| \mathbf{n} \|} = \frac{A(x_1 - x_0) + B(y_1 - y_0) + C(z_1 - z_0)}{\sqrt{A^2 + B^2 + C^2}} = \frac{|A x_1 + B y_1 + C z_1 + D|}{\sqrt{A^2 + B^2 + C^2}}
\]
where $\mathbf{b} = \overrightarrow{P_0P_1} = \langle x_1 - x_0, y_1 - y_0, z_1 - z_0 \rangle$.

\section{Cylinders and Quadric Surfaces}

\subsection{Cylinders}

A cylinder is a surface that consists of all lines that are parallel to a given line and pass through a given curve. The given line is called the \emph{generatrix} of the cylinder, and the given curve is called the \emph{directrix} of the cylinder.

\begin{example}
    Sketch the graph of the surface defined by the equation:
    \[
        z = x^2
    \]
\end{example}
\begin{solution}
    This equation represents a parabolic cylinder. For any fixed value of $y$, the cross-section in the $xz$-plane is a parabola defined by $z = x^2$. The surface extends infinitely along the $y$-axis, forming a cylinder-like shape. Consider the Figure \ref{fig:parabolic_cylinder} below, which illustrates the parabolic cylinder defined by the equation $z = x^2$. If we take cross-sections at different values of $y$, we obtain parabolas that open upwards in the $xz$-plane.
\end{solution}

\begin{figure}[h]
    \centering
    \tdplotsetmaincoords{70}{120}
	\begin{tikzpicture}[tdplot_main_coords]
		%%% Coordinate axis
		\draw[thick,-latex] (0,0,0) -- (2.5,0,0) node [below left] {\footnotesize$x$};
		\draw[dashed] (0,0,0) -- (-2.5,0,0);
		\draw[thick,-latex] (0,0,0) -- (0,4.5,0) node [right] {\footnotesize$y$};
		\draw[dashed] (0,0,0) -- (0,-2.5,0);
		\draw[thick] (0,0,0.0) -- (0,0,4.0);
		% The curves slicing the surface
		\draw[ocre,thick] plot[domain=-1.75:1.75,smooth,variable=\t] ({\t},{-2.0},{\t*\t});
		\draw[ocre,thick] plot[domain=-1.75:1.75,smooth,variable=\t] ({\t},{4.0},{\t*\t}); 
        \draw[ocre,thick] plot[domain=-1.75:1.75,smooth,variable=\t] ({\t},{0.0},{\t*\t});
		% The surface
		\foreach \y in {-1.99,-1.98,...,-0.01,0.01,0.02,...,3.99}{
			\draw[black!10,thick,opacity=0.2] plot[domain=-1.75:1.75,smooth,variable=\t] ({\t},{\y},{\t*\t}); 
		}
		%
		\node[above right] at (0,2.5,4.125) {$z = x^2$};
		% Last part of the z axis
		\draw[thick,-latex] (0,0,4.0) -- (0,0,4.5) node [above] {\footnotesize$z$};	
	\end{tikzpicture}
    \caption{Parabolic Cylinder of $z = x^2$}\label{fig:parabolic_cylinder}
\end{figure}

\begin{example}
    Sketch the graph of the surface defined by the equation:
    \[
        x^2 + y^2 = 1
    \]
\end{example}
\begin{solution}
    This equation represents a circular cylinder. For any fixed value of $z$, the cross-section in the $xy$-plane is a circle defined by $x^2 + y^2 = 1$. The surface extends infinitely along the $z$-axis, forming a cylinder-like shape. Consider the Figure \ref{fig:circular_cylinder} below, which illustrates the circular cylinder defined by the equation $x^2 + y^2 = 1$. If we take cross-sections at different values of $z$, we obtain circles in the $xy$-plane.
\end{solution}

\begin{figure}[h]
    \centering
    \tdplotsetmaincoords{70}{120}
    \begin{tikzpicture}[tdplot_main_coords]
		%%% Coordinate axis
		\draw[thick,-latex] (0,0,0) -- (2.5,0,0) node [below left] {\footnotesize$x$};
		\draw[dashed] (0,0,0) -- (-2.5,0,0);
		\draw[thick,-latex] (0,0,0) -- (0,2.5,0) node [right] {\footnotesize$y$};
		\draw[dashed] (0,0,0) -- (0,-2.5,0);
		\draw[thick] (0,0,0.0) -- (0,0,4.0);
		% The curves slicing the surface
		\draw[ocre,thick,opacity=0.5] (0,0,0) circle (1.0); 
		% The surface
		\foreach \z in {0.01,0.02,...,3.99}{
			\draw[black!10,thick,opacity=0.2] (0,0,\z) circle (1.0); 
		}
		% The curves slicing the surface
		\draw[ocre,thick,opacity=0.5] (0,0,4) circle (1.0);
		%

		\node[right] at (0,1,4) {$x^2 + y^2 = 1$};
		% Last part of the z axis
		\draw[thick,-latex] (0,0,4.0) -- (0,0,4.5) node [above] {\footnotesize$z$};	
    \end{tikzpicture}
    \caption{Circular Cylinder of $x^2 + y^2 = 1$}\label{fig:circular_cylinder}
\end{figure}

\subsection{Quadric Surfaces}
A quadric surface is a surface in three-dimensional space defined by a second-degree polynomial equation in three variables $x$, $y$, and $z$. The general form of a quadric surface equation is:
\[
    Ax^2 + By^2 + Cz^2 + Dxy + Eyz + Fxz + Gx + Hy + Iz + J = 0.
\]
By simple translation or rotations, it can be brought into one of the following forms:
\[
    Ax^2 + By^2 + Cz^2 + J = 0, \quad Ax^2 + By^2 + Iz = 0
\]

There are 6 kinds of quadric surfaces, as shown below:
\begin{figure}[h]

\centering

\begin{subfigure}{0.3\textwidth}
    \centering
    \begin{tikzpicture}
    \begin{axis}
    [view={135}{20},colormap={ocre}{
                color=(ocre) color=(ocre)
            },
    axis lines=center,axis on top,
    ticks=none,set layers=default,axis equal,
    xlabel={$x$}, ylabel={$y$}, zlabel={$z$},
    xlabel style={anchor=east},
    ylabel style={anchor=west},
    zlabel style={anchor=south},
    enlargelimits,
    tick align=inside,
    domain=0:2.00,
    samples=20, 
    z buffer=sort,
    ]
    \addplot3 [surf,draw=ocre!30,fill=white,samples=20, domain=-1:1, domain y=0:180, on layer=axis foreground] ({x}, {-2*cos(y)*sqrt(1-x^2)}, {-sin(y)*sqrt(1-x^2)});
    \addplot3 [surf,draw=ocre!30,fill=white,domain=-1:1,samples=20, domain y=0:180,on layer=axis foreground] ({x},{2*cos(y)*sqrt(1-x^2)},{sin(y)*sqrt(1-x^2)});
    \end{axis}
    \end{tikzpicture}
    \caption{Ellipsoid}
\end{subfigure}
\begin{subfigure}{0.3\textwidth}
    \centering
    \begin{tikzpicture}
    \begin{axis}
    [view={135}{20},colormap={ocre}{
                color=(ocre) color=(ocre)
            },
    axis lines=center,axis on top,
    ticks=none,set layers=default,axis equal,
    xlabel={$x$}, ylabel={$y$}, zlabel={$z$},
    xlabel style={anchor=east},
    ylabel style={anchor=west},
    zlabel style={anchor=south},
    enlargelimits,
    tick align=inside,
    domain=0:2.00,
    samples=20, 
    z buffer=sort,
    ]
    \addplot3 [surf,draw=ocre!30,fill=white,samples=20, domain=0:1, domain y=-180:180,on layer=axis foreground] ({-2*sqrt(x)*cos(y)},{-sqrt(x)*sin(y)},{x});
    \end{axis}
    \end{tikzpicture}
    \caption{Elliptic Paraboloid}
\end{subfigure}
\begin{subfigure}{0.3\textwidth}
    \centering
    \begin{tikzpicture}
    \begin{axis}
    [view={135}{20},colormap={ocre}{
                color=(ocre) color=(ocre)
            },
    axis lines=center,axis on top,
    ticks=none,set layers=default,axis equal,
    xlabel={$x$}, ylabel={$y$}, zlabel={$z$},
    xlabel style={anchor=east},
    ylabel style={anchor=west},
    zlabel style={anchor=south},
    enlargelimits,
    tick align=inside,
    domain=0:2.00,
    samples=20, 
    z buffer=sort,
    ]
    \addplot3 [surf,draw=ocre!30,fill=white,samples=20, domain=-0.5:0.5, domain y=-0.5:0.5, on layer=axis foreground] {x^2 - 4*y^2};
    \end{axis}
    \end{tikzpicture}
    \caption{Hyperbolic Paraboloid}
\end{subfigure}
\begin{subfigure}{0.3\textwidth}
    \centering
    \begin{tikzpicture}
    \begin{axis}
    [view={135}{20},colormap={ocre}{
                color=(ocre) color=(ocre)
            },
    axis lines=center,axis on top,
    ticks=none,set layers=default,axis equal,
    xlabel={$x$}, ylabel={$y$}, zlabel={$z$},
    xlabel style={anchor=east},
    ylabel style={anchor=west},
    zlabel style={anchor=south},
    enlargelimits,
    tick align=inside,
    domain=0:2.00,
    samples=20, 
    z buffer=sort,
    ]
    \addplot3 [surf,draw=ocre!30,fill=white,samples=20, domain=-1:1, domain y=-180:180,on layer=axis foreground] ({-2*(x)*cos(y)},{-(x)*sin(y)},{x});
    \end{axis}
    \end{tikzpicture}
    \caption{Cone}
\end{subfigure}
\begin{subfigure}{0.3\textwidth}
    \centering
    \begin{tikzpicture}
    \begin{axis}
    [view={135}{20},colormap={ocre}{
                color=(ocre) color=(ocre)
            },
    axis lines=center,axis on top,
    ticks=none,set layers=default,axis equal,
    xlabel={$x$}, ylabel={$y$}, zlabel={$z$},
    xlabel style={anchor=east},
    ylabel style={anchor=west},
    zlabel style={anchor=south},
    enlargelimits,
    tick align=inside,
    domain=0:2.00,
    samples=20, 
    z buffer=sort,
    ]
    \addplot3 [surf,draw=ocre!30,fill=white,samples=20, domain=-1:1, domain y=-180:180,on layer=axis foreground] ({2*sqrt(1+x^2)*cos(y)},{sqrt(1+x^2)*sin(y)},{x});
    \end{axis}
    \end{tikzpicture}
    \caption{Hyperboloid of One Sheet}
\end{subfigure}
\begin{subfigure}{0.3\textwidth}
    \centering
    \begin{tikzpicture}
    \begin{axis}
    [view={135}{20},colormap={ocre}{
                color=(ocre) color=(ocre)
            },
    axis lines=center,axis on top,
    ticks=none,set layers=default,axis equal,
    xlabel={$x$}, ylabel={$y$}, zlabel={$z$},
    xlabel style={anchor=east},
    ylabel style={anchor=west},
    zlabel style={anchor=south},
    enlargelimits,
    tick align=inside,
    domain=0:2.00,
    samples=20, 
    z buffer=sort,
    ]
    \addplot3 [surf,draw=ocre!30,fill=white,samples=20, domain=1:2, domain y=-180:180,on layer=axis foreground] ({2*sqrt(x^2 - 1)*cos(y)},{sqrt(x^2 - 1)*sin(y)},{x});
    \addplot3 [surf,draw=ocre!30,fill=white,samples=20, domain=-2:-1, domain y=-180:180,on layer=axis foreground] ({2*sqrt(x^2 - 1)*cos(y)},{sqrt(x^2 - 1)*sin(y)},{x});
    \end{axis}
    \end{tikzpicture}
    \caption{Hyperboloid of Two Sheets}
\end{subfigure}

\end{figure}

\section{Vector Functions}

A vector function is a function that takes one or more variables and returns a vector. In three-dimensional space, a vector function can be represented as:
\[
    \mathbf{r}(t) = \langle x(t), y(t), z(t) \rangle
\]

The limit of the vector function $\mathbf{r}(t)$ as $t$ approaches $t_0$ is defined as:
\[
    \lim_{t \to t_0} \mathbf{r}(t) = \left\langle \lim_{t \to t_0} x(t), \lim_{t \to t_0} y(t), \lim_{t \to t_0} z(t) \right\rangle
\]

The derivatives of the vector function $\mathbf{r}(t)$ is defined as:
\[
    \dfrac{d\mathbf{r}}{dt} = \mathbf{r}'(t) = \lim_{h \to 0} \frac{\mathbf{r}(t + h) - \mathbf{r}(t)}{h} = \langle x'(t), y'(t), z'(t) \rangle
\]
There are some properties for derivatives of vector functions:
\begin{itemize}
    \item $\dfrac{d}{dt} [\mathbf{u}(t) + \mathbf{v}(t)] = \mathbf{u}'(t) + \mathbf{v}'(t)$
    \item $\dfrac{d}{dt} [c\mathbf{u}(t)] = c\mathbf{u}'(t)$
    \item $\dfrac{d}{dt} [f(t) \mathbf{u}(t)] = f'(t) \mathbf{u}(t) + f(t) \mathbf{u}'(t)$
    \item $\dfrac{d}{dt} [\mathbf{u}(t) \cdot \mathbf{v}(t)] = \mathbf{u}'(t) \cdot \mathbf{v}(t) + \mathbf{u}(t) \cdot \mathbf{v}'(t)$
    \item $\dfrac{d}{dt} [\mathbf{u}(t) \times \mathbf{v}(t)] = \mathbf{u}'(t) \times \mathbf{v}(t) + \mathbf{u}(t) \times \mathbf{v}'(t)$
    \item $\dfrac{d}{dt} [\mathbf{u}(f(t))] = f'(t) \mathbf{u}'(f(t))$
\end{itemize}

The definite integral of vector functions $\mathbf{r}(t)$ from $a$ to $b$ is defined as:
\[
    \int_a^b \mathbf{r}(t) dt = \left\langle \int_a^b x(t) dt, \int_a^b y(t) dt, \int_a^b z(t) dt \right\rangle
\]

Arc length:
\[
    L = \int_a^b \| \mathbf{r}'(t) \| dt = \int_a^b \sqrt{(x'(t))^2 + (y'(t))^2 + (z'(t))^2} dt
\]

Arc length parametrisation:

Given a curve $\mathbf{r}(t)$, compute the integral:
\[
    s = s(t) = \int_a^t \| \mathbf{r}'(\tau) \| d\tau
\]
Then express $t$ as a function of $s$, i.e., $t = t(s)$. Finally replace all $t$ in $\mathbf{r}(t)$ as $\mathbf{r}(t(s))$, a function in terms of $s$.

\begin{example}
    Find the arc-length parametrisation of the curve:
    \[
        \mathbf{r}(t) = \langle \cos t, \sin t, t \rangle, \qquad t \in [0, 2\pi].
    \]
\end{example}
\begin{solution}
    We have:
    \[
        \| \mathbf{r}'(t) \| = \sqrt{(-\sin t)^2 + (\cos t)^2 + 1^2} = \sqrt{2}.
    \]
    So,
    \[
        s = \int_0^t \sqrt{2} d\tau = \sqrt{2} t.
    \]
    Express $t$ in terms of $s$, we get $t = \frac{s}{\sqrt{2}}$. Replace all $t$'s in $\mathbf{r}(t)$, we have the arc-length parametrisation:
    \[
        \tilde{\mathbf{r}}(s) = \left\langle \cos\left(\frac{s}{\sqrt{2}}\right), \sin\left(\frac{s}{\sqrt{2}}\right), \frac{s}{\sqrt{2}} \right\rangle, \qquad s \in [0, 2\pi\sqrt{2}].
    \]
\end{solution}


\chapter{Partial Derivatives}


\chapter{Multiple Integrals}
\section{Partial Integration}
    We have learnt how to calculate the integration of a function in single variable. Now, we extends our knowledge to functions in several variables. One should understand that the partial integration is the reverse process of partial differentiation.
    \\\indent Define a function $f(x,y):\R^2\to\R$, we have 
    \[\int f dx \quad \text{and}\quad \int f dy\]
    Note that the above integrals are not the same as the single variable integration since $f$ is a function of two variables. The above integrals are called \textbf{partial integrals}. In general, we have 
    \\Given a function $f:\R^n \to \R$
    \[\int f dx_1,\quad \int f dx_2,\quad \dots ,\quad \int f dx_n\]
    where $x_1,x_2,\ldots,x_n$ are the variables of integration.
    \begin{example}
        Given a function $f(x,y)=x^2y+3xy^2$, find $\int f dx$ and $\int f dy$.
        \begin{solution}
            Notice that when we integrate with respect to $x$, we treat $y$ as a constant. So as the other way around. Thus, 
            \[
            \begin{split}
                \int x^2y+3xy^2 dx&= \frac{y}{3}x^3+\frac{3y^2}{2}x^2 +C(y)\\
                \int x^2y+3xy^2 dy&= \frac{x^2}{2}y^2+xy^3 +C(x)
            \end{split}
            \]
        \end{solution}
        The integration constants $C(y)$ and $C(x)$ in this case are functions in $x$ and $y$ rather than just a constant number.
    \end{example}

    \begin{example}
        Given $f(x,y)=ye^{xy^2}$, find $\int f dx$ and $\int f dy$.
        \begin{solution}
            \[
            \begin{split}
                \int ye^{xy^2} dx&= \frac{e^{xy^2}}{y}+C(y)\\
                \int ye^{xy^2} dy&= \frac{1}{2x}e^{xy^2}+C(x)
            \end{split}
            \]
            We can substitute $u=xy^2$, then $du=y^2dx$ and $du=2xy dy$ to compute the integrals.
        \end{solution}
    \end{example}

    \section{Definite integration}
    The concept here is similar to the single variable definite integration. We define the definite partial integral of $f(x,y)$ with respect to $x$ from $a$ to $b$ as
    \[\int_a^b f(x,y) dx =\int_{x=a}^{x=b} f(x,y) dx= F(b,y)-F(a,y)\]
    Similarly, we may define the definite partial integral of $f(x,y)$ with respect to $y$ from $c$ to $d$ as
    \[\int_c^d f(x,y) dy =\int_{y=c}^{y=d} f(x,y) dy= G(x,d)-G(x,c)\]
    Note that $y$ and $x$ are treated as constants in the above two definitions respectively.

    \begin{example}
        Given $f(x,y)=x^2y+3xy^2$, find $\int_1^3 f(x,y) dx$ and $\int_1^3 f(x,y) dy$.
        \begin{solution}
            \[
            \begin{split}
                \int_1^3 (x^2y+3xy^2) dx&= \left[\frac{y}{3}x^3+\frac{3y^2}{2}x^2\right]_{x=1}^{x=3}= \frac{26}{3}y + 12y^2\\
                \int_1^3 (x^2y+3xy^2) dy&= \left[\frac{x^2}{2}y^2+xy^3\right]_{y=1}^{y=3}= 4x^2 + 26x
            \end{split}
            \]
        \end{solution}
    \end{example}

    \section{Double Integrals}
    A double integral is an extension of the single variable definite integral to functions of two variables. It is used to calculate the volume under a surface defined by a function $f(x,y)$ over a rectangular region in the $xy$-plane. The double integral of $f(x,y)$ over the rectangular region $R = [a,b] \times [c,d]$ is denoted as:
    \[\iint_R f(x,y) dA = \int_a^b \int_c^d f(x,y) dy dx\]
    where $dA$ represents a small area element on the $xy$-plane.



\chapter{Vector Calculus}

\end{document}